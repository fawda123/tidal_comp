\documentclass[letterpaper,12pt]{article}\usepackage[]{graphicx}\usepackage[]{color}
%% maxwidth is the original width if it is less than linewidth
%% otherwise use linewidth (to make sure the graphics do not exceed the margin)
\makeatletter
\def\maxwidth{ %
  \ifdim\Gin@nat@width>\linewidth
    \linewidth
  \else
    \Gin@nat@width
  \fi
}
\makeatother

\definecolor{fgcolor}{rgb}{0.345, 0.345, 0.345}
\newcommand{\hlnum}[1]{\textcolor[rgb]{0.686,0.059,0.569}{#1}}%
\newcommand{\hlstr}[1]{\textcolor[rgb]{0.192,0.494,0.8}{#1}}%
\newcommand{\hlcom}[1]{\textcolor[rgb]{0.678,0.584,0.686}{\textit{#1}}}%
\newcommand{\hlopt}[1]{\textcolor[rgb]{0,0,0}{#1}}%
\newcommand{\hlstd}[1]{\textcolor[rgb]{0.345,0.345,0.345}{#1}}%
\newcommand{\hlkwa}[1]{\textcolor[rgb]{0.161,0.373,0.58}{\textbf{#1}}}%
\newcommand{\hlkwb}[1]{\textcolor[rgb]{0.69,0.353,0.396}{#1}}%
\newcommand{\hlkwc}[1]{\textcolor[rgb]{0.333,0.667,0.333}{#1}}%
\newcommand{\hlkwd}[1]{\textcolor[rgb]{0.737,0.353,0.396}{\textbf{#1}}}%

\usepackage{framed}
\makeatletter
\newenvironment{kframe}{%
 \def\at@end@of@kframe{}%
 \ifinner\ifhmode%
  \def\at@end@of@kframe{\end{minipage}}%
  \begin{minipage}{\columnwidth}%
 \fi\fi%
 \def\FrameCommand##1{\hskip\@totalleftmargin \hskip-\fboxsep
 \colorbox{shadecolor}{##1}\hskip-\fboxsep
     % There is no \\@totalrightmargin, so:
     \hskip-\linewidth \hskip-\@totalleftmargin \hskip\columnwidth}%
 \MakeFramed {\advance\hsize-\width
   \@totalleftmargin\z@ \linewidth\hsize
   \@setminipage}}%
 {\par\unskip\endMakeFramed%
 \at@end@of@kframe}
\makeatother

\definecolor{shadecolor}{rgb}{.97, .97, .97}
\definecolor{messagecolor}{rgb}{0, 0, 0}
\definecolor{warningcolor}{rgb}{1, 0, 1}
\definecolor{errorcolor}{rgb}{1, 0, 0}
\newenvironment{knitrout}{}{} % an empty environment to be redefined in TeX

\usepackage{alltt}
\usepackage[top=1in,bottom=1in,left=1in,right=1in]{geometry}
\usepackage{setspace}
\usepackage[colorlinks=true,urlcolor=blue,citecolor=blue,linkcolor=blue]{hyperref}
\usepackage{indentfirst}
\usepackage{multirow}
\usepackage{booktabs}
\usepackage[final]{animate}
\usepackage{graphicx}
\usepackage{verbatim}
\usepackage{rotating}
\usepackage{tabularx}
\usepackage{array}
\usepackage{subfig} 
\usepackage[noae]{Sweave}
\usepackage{cleveref}
\usepackage[figureposition=bottom]{caption}
\usepackage{paralist}
\usepackage{acronym}
\usepackage{outlines}
\usepackage{pdflscape}

% knitr options


\IfFileExists{upquote.sty}{\usepackage{upquote}}{}
\begin{document}

\setlength{\parskip}{5mm}
\setlength{\parindent}{0in}

\title{Summary of data processing for Patuxent River Estuary}
\author{Marcus W. Beck, Rebecca Murphy}
\maketitle

The following describes additional processing of raw station data for the Patuxent River Estuary. Note that the raw data file (`PAX\_TRIB...') was edited manually to remove station TF1.0 and to change the chlorophyll value for TF1.4 on 11/28/1988 from 2.99 to 5.98.  Briefs descriptions of each step are provided.

The following raw data files were imported:
\begin{enumerate}
\item \texttt{PAX\_TRIB\_CHLAandSALINITY\_85to14.csv}: chlorophyll and salinity data for all stations in the Patuxent River from 1985 to 2014 (from R. Murphy)
\item \texttt{PAX\_station\_info.csv}: metadata for each station including lat/lon, salinity zone, etc. (from R. Murphy)
\end{enumerate}

The data were first imported into R.
\begin{kframe}
\begin{alltt}
\hlcom{# code for processing raw data, see email from R. Murphy on 3/13/15}
\hlcom{# created March 2015, M. Beck}

\hlcom{## packages to use}
\hlcom{# this is just to load dplyr, ggplot2}
\hlstd{devtools}\hlopt{::}\hlkwd{load_all}\hlstd{(}\hlstr{'M:/docs/tidal_comp/TidalComp'}\hlstd{)}

\hlcom{## import}

\hlcom{# meta}
\hlstd{pax_meta} \hlkwb{<-} \hlkwd{system.file}\hlstd{(}\hlstr{'PAX_station_info.csv'}\hlstd{,} \hlkwc{package} \hlstd{=} \hlstr{'TidalComp'}\hlstd{)}
\hlstd{pax_meta} \hlkwb{<-} \hlkwd{read.csv}\hlstd{(pax_meta,} \hlkwc{header} \hlstd{=} \hlnum{TRUE}\hlstd{,}
  \hlkwc{stringsAsFactors} \hlstd{=} \hlnum{FALSE}\hlstd{)}

\hlcom{# data}
\hlstd{pax_data} \hlkwb{<-} \hlkwd{system.file}\hlstd{(}\hlstr{'PAX_TRIB_CHLAandSALINITY_85TO14.csv'}\hlstd{,}
  \hlkwc{package} \hlstd{=} \hlstr{'TidalComp'}\hlstd{)}
\hlstd{pax_data} \hlkwb{<-} \hlkwd{read.csv}\hlstd{(pax_data,} \hlkwc{header} \hlstd{=} \hlnum{TRUE}\hlstd{,}
  \hlkwc{stringsAsFactors} \hlstd{=} \hlnum{FALSE}\hlstd{)}

\hlcom{# reorder STATION variable along trib axis}
\hlstd{stats} \hlkwb{<-} \hlkwd{c}\hlstd{(}\hlstr{'TF1.3'}\hlstd{,} \hlstr{'TF1.4'}\hlstd{,} \hlstr{'TF1.5'}\hlstd{,} \hlstr{'TF1.6'}\hlstd{,} \hlstr{'TF1.7'}\hlstd{,}
  \hlstr{'RET1.1'}\hlstd{,} \hlstr{'LE1.1'}\hlstd{,} \hlstr{'LE1.2'}\hlstd{,} \hlstr{'LE1.3'}\hlstd{,} \hlstr{'LE1.4'}\hlstd{)}
\hlstd{pax_data}\hlopt{$}\hlstd{STATION} \hlkwb{<-} \hlkwd{factor}\hlstd{(pax_data}\hlopt{$}\hlstd{STATION,} \hlkwc{level} \hlstd{= stats)}
\end{alltt}
\end{kframe}

Salinity data were vertically-integrated for each  unique date. The integration function averaged all salinity values after interpolating from the surface to the maximum depth.  Salinity values at the most shallow and deepest sampling depth were repeated for zero depth and maximum depths, respectively, to bound the interpolations within the range of the data.  A similar process for vertically-integrating salinity across depth values was used for the remaining station data.
\begin{kframe}
\begin{alltt}
\hlcom{##}
\hlcom{# get vertically integrated salinity}

\hlcom{# vertical integration by date, station}
\hlstd{int_fun} \hlkwb{<-} \hlkwa{function}\hlstd{(}\hlkwc{TOTAL_DEPTH}\hlstd{,} \hlkwc{DEPTH}\hlstd{,} \hlkwc{AvgValue}\hlstd{)\{}

  \hlkwa{if}\hlstd{(}\hlkwd{length}\hlstd{(}\hlkwd{na.omit}\hlstd{(AvgValue))} \hlopt{<} \hlnum{2} \hlstd{)} \hlkwd{return}\hlstd{(}\hlkwd{na.omit}\hlstd{(AvgValue))}

  \hlcom{# setup for interpolation}
  \hlstd{max_depths} \hlkwb{<-} \hlkwd{mean}\hlstd{(}\hlkwd{unique}\hlstd{(TOTAL_DEPTH),} \hlkwc{na.rm} \hlstd{=} \hlnum{TRUE}\hlstd{)}
  \hlstd{depths} \hlkwb{<-} \hlkwd{c}\hlstd{(}\hlnum{0}\hlstd{, DEPTH, max_depths)}
  \hlstd{vals} \hlkwb{<-} \hlkwd{c}\hlstd{(AvgValue[}\hlnum{1}\hlstd{], AvgValue, AvgValue[}\hlkwd{length}\hlstd{(AvgValue)])}

  \hlcom{# only interpolate if > 1 salinity value}
  \hlstd{out} \hlkwb{<-} \hlkwd{mean}\hlstd{(}\hlkwd{approx}\hlstd{(depths, vals)}\hlopt{$}\hlstd{y)}

  \hlkwd{return}\hlstd{(out)}

  \hlstd{\}}

\hlcom{# process}
\hlcom{# note that there are no 'PROBLEM' values, lab and method do not change}
\hlstd{sal_tmp} \hlkwb{<-} \hlkwd{filter}\hlstd{(pax_data, PARAMETER} \hlopt{==} \hlstr{'SALINITY'}\hlstd{)} \hlopt
  \hlkwd{mutate}\hlstd{(}\hlkwc{date} \hlstd{=} \hlkwd{as.Date}\hlstd{(date,} \hlkwc{format} \hlstd{=} \hlstr{'%m/%d/%Y'}\hlstd{))} \hlopt
  \hlkwd{group_by}\hlstd{(date, STATION)} \hlopt
  \hlkwd{summarize}\hlstd{(}\hlkwc{sal} \hlstd{=} \hlkwd{int_fun}\hlstd{(TOTAL_DEPTH, DEPTH, AvgValue))}
\end{alltt}
\end{kframe}

Chlorophyll values at each station were retained only for surface samples and no `problem' codes.  Chlorophyll were also transformed by the natural-log.  
\begin{kframe}
\begin{alltt}
\hlcom{##}
\hlcom{# get only surface estimates for chlorophyll}
\hlcom{# remove those w/ problem codes}
\hlstd{chl_tmp} \hlkwb{<-} \hlkwd{filter}\hlstd{(pax_data,}
  \hlstd{PARAMETER} \hlopt{==} \hlstr{'CHLA'} \hlopt{&} \hlstd{LAYER} \hlopt{==} \hlstr{'S'}\hlopt{&} \hlstd{PROBLEM} \hlopt{==} \hlstr{''}
  \hlstd{)} \hlopt
  \hlkwd{mutate}\hlstd{(}\hlkwc{lnchla} \hlstd{=} \hlkwd{log}\hlstd{(AvgValue))} \hlopt
  \hlkwd{mutate}\hlstd{(}\hlkwc{date} \hlstd{=} \hlkwd{as.Date}\hlstd{(date,} \hlkwc{format} \hlstd{=} \hlstr{'%m/%d/%Y'}\hlstd{))} \hlopt
  \hlkwd{select}\hlstd{(date, STATION, lnchla)}
\end{alltt}
\end{kframe}

Some plots of the raw data.
\begin{kframe}
\begin{alltt}
\hlcom{## }
\hlcom{# merge chl and salinity data, then plot}

\hlstd{pax_data} \hlkwb{<-} \hlkwd{full_join}\hlstd{(chl_tmp, sal_tmp,} \hlkwc{by} \hlstd{=} \hlkwd{c}\hlstd{(}\hlstr{'date'}\hlstd{,} \hlstr{'STATION'}\hlstd{))}

\hlkwd{ggplot}\hlstd{(pax_data,} \hlkwd{aes}\hlstd{(}\hlkwc{x} \hlstd{= date,} \hlkwc{y} \hlstd{= sal,} \hlkwc{group} \hlstd{= STATION))} \hlopt{+}
  \hlkwd{geom_line}\hlstd{()} \hlopt{+}
  \hlkwd{theme_classic}\hlstd{()} \hlopt{+}
  \hlkwd{facet_wrap}\hlstd{(}\hlopt{~} \hlstd{STATION,} \hlkwc{ncol} \hlstd{=} \hlnum{3}\hlstd{)}

\hlkwd{ggplot}\hlstd{(pax_data,} \hlkwd{aes}\hlstd{(}\hlkwc{x} \hlstd{= date,} \hlkwc{y} \hlstd{= lnchla,} \hlkwc{group} \hlstd{= STATION))} \hlopt{+}
  \hlkwd{geom_line}\hlstd{()} \hlopt{+}
  \hlkwd{theme_classic}\hlstd{()} \hlopt{+}
  \hlkwd{facet_wrap}\hlstd{(}\hlopt{~} \hlstd{STATION,} \hlkwc{ncol} \hlstd{=} \hlnum{3}\hlstd{)}

\hlcom{# save the data}
\hlstd{save_path} \hlkwb{<-} \hlkwd{gsub}\hlstd{(}\hlstr{'text$'}\hlstd{,} \hlstr{'data'}\hlstd{,} \hlkwd{getwd}\hlstd{())}
\hlkwd{save}\hlstd{(pax_data,} \hlkwc{file} \hlstd{=} \hlkwd{paste0}\hlstd{(save_path,} \hlstr{'/pax_data.RData'}\hlstd{))}
\end{alltt}
\end{kframe}

{\centering \includegraphics[width=\maxwidth]{figs/unnamed-chunk-41} 
\includegraphics[width=\maxwidth]{figs/unnamed-chunk-42} 

}




\end{document}
